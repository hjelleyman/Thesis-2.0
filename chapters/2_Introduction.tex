\documentclass[../main.tex]{subfiles}

\begin{document}

\chapter{Introduction}

Ice in Antarctica is critical for our understanding of global climate patterns and climate change. It has a high albedo and stores massive ammounts of both energy and water, being intrinsicly linked with both the water cycle and energy balance for the Southern Hemisphere. Sea ice melts and reforms around the continent each year, trapping extra C0$_2$ in the Ocean. It is also shown to affect the salinity of the ocean and ocean circulations. 
The extent of ice is additionally linked to the state of the climate in Antarctica, and therefore by extension, the state of our global climate. Understanding the processes which have driven its behaviours and which could potentially drive the trends and variability we see in both land and sea ice in Antarctica in the future is therefore important for our understanding of the changing climate of Earth.
For this thesis, the key question we are attempting to ask is this:

\begin{quote}
	\textit{What drives the trends and variability we see in Antarctic Ice?}
\end{quote}

In order to answer this question we will make use of statistical techniques such as linear regressions and Pearson correlations to determine the relationships which we can observe between ice in Antarctica, different environmental variables such as temperature, Ozone concentration, and surface pressure, and climatic patterns which are known to drive global and southern hemisphere climate such as the El Ni\~no Southern Oscillation and the Southern Anular Mode.

This introduction will include details about the data we have used in this project, followed by a brief overview of the analysis carried out on the data and a discussion of our key findings. The rest of the thesis will cover the analysis and results in more detail alongside a literature review which should introduce the key ideas which you need to know to understand the state of contempoary research on ice in Antarctica and its relationship with global climate.


\pagebreak
\section{Data used in this project}

\subsection{Antarctic ice data}
\subsubsection*{NSIDC}
\subsubsection*{GRACE-FO}

\subsection{Climatic Indices}
\subsubsection*{SAM}
\subsubsection*{ENSO}
\subsubsection*{DMI}
\subsubsection*{IPO}
\subsubsection*{Non-Pacific Climate Modes}

\subsection{ECMWF ERA5 reanalysis}

\section{Planning the analysis}

\section{Our key findings}
\end{document}