
\section{Environmental variables}

\subsection{Describing the environmental variables}

For this project we use ERA5 reanalysis data produced by ECMWF \textcolor{red}{cite this} as our source of data for the environmental variables listed below.

\begin{enumerate}
	\item Skin Temperature [SKT]
	\item 2 metre Temperature [T2M]
	\item Sea Surface Temperature [SST]
\end{enumerate}

This data comes in a regular \textcolor{red}{check resolution} 0.25 degrees grid which covers the entire globe. For our analysis which involves looking at weather in Antarctica we projected this data onto the standard South Polar Stereographic projection as defined by NSIDC \textcolor{red}{link this here}.
This means that the data we use for analysis is interpolated which lowers the quality slightly, but negilibly in compared to the utillity we gain from having it in the same projection and coordinates as our other sources of data.

Next we should consider the different variables we used from this dataset. Listed above, we will now go through them one by one and explain how they are calculated and what this may mean for the project and any results we may generate.

For a full technical description of how any and all variables are calculated for the ERA5 reanalysis you can refer to \textcolor{red}{cite report}. We will briefly summarise the most relevant points here for convenience of the reader.

\subsubsection*{Skin Temperature [SKT]}

The skin temperature is the theoretical temperature that is required to satisfy the surface energy balance. It represents the temperature of the uppermost surface layer, which has no heat capacity and so can respond instantaneously to changes in surface fluxes. Skin temperature is calculated differently over land and sea because of the difference in dynamics in each type of terrain.

In the presense of sea ice, skin temperature is taken as the temperature on the upper surface of the ice.

\subsubsection*{2 metre Temperature [T2M]}

2 metre temperature is the temperature measurement which is most often used in literature because of its regularity. 2m temperature is calculated by interpolating between the lowest model level \textcolor{red}{check what pressure this is} and the Earth's surface, taking account of the atmospheric conditions.

\subsubsection*{Sea Surface Temperature [SST]}

SST is usually calculated directly from satilite data, but in the absense of that it is calculated using data from NEMO forecasts 
\textcolor{red}{cite this}.
Sea surface temperature is masked to be only over sea and as such we can only use it when doing calculations over the ocean. Nonetheless it is still useful.
When the ocean is exposed this will be very similar to skin temperature, however when sea ice is found, SST is usually set to -1.65$^\circ$C, the freezing point for salt water. 

\subsection{Example plots}


\subsection{Validation}


